\documentclass[11pt]{article}
\usepackage{acl2012}
\usepackage{times}
\usepackage{latexsym}
\usepackage{amsmath}
\usepackage{multirow}
\usepackage{url}
\DeclareMathOperator*{\argmax}{arg\,max}
\setlength\titlebox{6.5cm}    % Expanding the titlebox

\title{IBM Model 1}

\author{Anouk Visser \\
  Affiliation / Address line 1 \\
  {\tt anouk.visser@student.uva.nl} \\\And
  R\'emi de Zoeten \\
  Affiliation / Address line 1 \\
  {\tt email@domain} \\}

\date{}

\begin{document}
\maketitle

\section{Introduction}
One solution to machine translation is to find translations of words in the sentence. IBM Model 1 is an example of such a word-based model. In this report we present our implementation of IBM Model 1 and its Expectation-Maximization (EM) training. In section \ref{IBM} we will give a little background on IBM Model 1, in section \ref{em} we lay out the EM formulas. In section \ref{own} we propose a simple improvement to IBM Model 1 reducing its assumptions. Finally, we present our results in section \ref{results}.


\section{IBM Model I}
\label{IBM}
IBM model 1 is the first and least complex in the series of IBM translation models. It is based only on word level translation probabilities that are derived from a training corpus. The model only defines translation probabilities as $P(f|e)$ where $f$ is a french word and $e$ is an english word and does not look at word context or sentence structure. The $P(f|e)$ is obtained using the EM algorithm described in \ref{em}. IBM model 1 does not address word re-ordering or sentence structure.

\section{Expectation Maximization Training Formula}
\label{em}
The Expectation-Maximization (EM) algorithm is used to iteratively re-estimate the probability of $P(f|e)$, converging on every iteration. The estimated probabilities $P(f|e)$ are initialized uniformly and are stored in a translation table $T$. By first applying the current model to the data (starting with uniform) we can get an expectation (E-step), we can use this information to learn the model from the actual data (M-step).\\\\
The E-step is performed separately for every sentence pair. We iterate over all possible alignments to find the probability of an alignment given the source and the target by looking up the probability for the required translations (for that specific alignment) in the translation table and normalizing this: 
$$P_t(a|f, e) = \prod\limits_{j=1}^{m} (\frac{t(f_j|e_{a_{j}})}{\sum\limits_{i=0}^{l}t(f_j|e_i)})$$
The M-step re-estimates the translation probabilities by weighing the counts of the number of times an alignment occurred by its probability and normalizing this. 
$$t(f|e) = \frac{\sum\limits_{(\textbf{e}, \textbf{f})} c(e|f; \textbf{e}, \textbf{f})}{\sum\limits_{e}\sum\limits_{(\textbf{e}, \textbf{f})} c(e|f; \textbf{e}, \textbf{f})}$$

\section{Improvements}
\label{own}

\section{Results}
\label{results}

\section{Conclusion}

\begin{thebibliography}{}

\bibitem[\protect\citename{Aho and Ullman}1972]{Aho:72}
Alfred~V. Aho and Jeffrey~D. Ullman.
\newblock 1972.
\newblock {\em The Theory of Parsing, Translation and Compiling}, volume~1.
\newblock Prentice-{Hall}, Englewood Cliffs, NJ.


\end{thebibliography}

\end{document}
